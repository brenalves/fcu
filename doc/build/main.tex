%----------------------------------------------------------------------------------
% Exemplo do uso da classe tcc.cls. Veja o arquivo .cls
% para mais detalhes e instruções.
%----------------------------------------------------------------------------------

% Seleção de idioma da monografia. Por enquanto as únicas opções
% suportadas são 'portuguese' e 'english'
% Para impressão em frente e verso, use a opção 'twoside'. Da
% mesma forma, use 'oneside' para impressão em um lado apenas.
\documentclass[english,oneside]{UFFStex}
%\documentclass[english,oneside]{UFFStex}

%----------------------------------------------------------------
% Coloque seus pacotes abaixo.
%
% Obs.: muitos pacotes de uso comum do LaTeX, como amsmath,
% geometry e url já são automaticamente incluídos pela classe
% (veja o arquivo .cls). Isso torna obrigatória a presença destes
% no sistema para o uso desta classe, mas ao mesmo tempo o uso se
% torna mais simples.  Recomendo a instalação da versão mais
% recente da distribuição TeXLive (para Windows e UNIXes):
% www.tug.org/texlive/
%
% Pacotes e opções já incluídas automaticamente:
%
% \RequirePackage[T1]{fontenc}[2005/09/27]
% \RequirePackage[utf8x]{inputenc}[2008/03/30]
% \RequirePackage[english,brazil]{babel}[2008/07/06]
% \RequirePackage[a4paper]{geometry}[2010/09/12]
% \RequirePackage{textcomp}[2005/09/27]
% \RequirePackage{lmodern}[2009/10/30]
% \RequirePackage{indentfirst}[1995/11/23]
% \RequirePackage{setspace}[2000/12/01]
% \RequirePackage{textcase}[2004/10/07]
% \RequirePackage{float}[2001/11/08]
% \RequirePackage{amsmath}[2000/07/18]
% \RequirePackage{amssymb}[2009/06/22]
% \RequirePackage{amsfonts}[2009/06/22]
% \RequirePackage{url}
% \RequirePackage[table]{xcolor}[2007/01/21]
%\RequirePackage{lastpage}
%\RequirePackage{xparse}
%\RequirePackage{xstring}
%\RequirePackage{listofitems}
%----------------------------------------------------------------
% Para inserção de figuras.
\usepackage{graphicx}
% Utilize a opção 'pdftex' se você estiver usando o pdflatex (que
% permite figuras em formatos como .jpg ou .png)
%\usepackage[pdftex]{graphicx}

% Para tabelas com elementos ocupando mais de uma linha
\usepackage{multirow}
% Para frações na mesma linha (ex. ⅓).
\usepackage{nicefrac}
% Para inserir figuras lado a lado.
 \usepackage{subfigure}
% Para formatar algoritmos.
% A opção [algo2e] é necessária para evitar conflitos
% com as definições da classe.
% https://linorg.usp.br/CTAN/macros/latex/contrib/algorithm2e/doc/algorithm2e.pdf
\usepackage{algorithmic}
\usepackage[portuguese,algo2e,boxruled]{algorithm2e}
% Hyperlink em capitulos, seções, citações
\usepackage[hidelinks]{hyperref}	
% Bibliografia e citações padrão ABNT
%https://mirrors.ibiblio.org/CTAN/macros/latex/contrib/abntex2/doc/abntex2cite-alf.pdf
% Comandos: \citeonline{ref} \citeauthor{ref} \citeyear{ref}
%\usepackage[num]{abntex2cite}	% numérica
\usepackage[alf]{abntex2cite}	% alfabetica
% Referência cruzada que já coloca o tipo (Tabela, Figura, etc)
% a opção "brazilian" é para colocar em portugues (retirar a opção 
% quando escrever o texto em ingles/english
\usepackage[brazilian]{cleveref} 
% Pacote para revisão de texto
% alguns comandos de revisão:
% \alert{texto entre chaves}  \highlight{texto entre chaves}
% \remove{texto a ser removido} 
% \replace{texto original}{texto novo}
% \comment{texto original}{comentario adicionado}
% precisa ajustar a margem para evitar warning
\setlength {\marginparwidth }{2cm}
\usepackage{easyReview} 
\usepackage{tabularx}
\usepackage{pgfgantt}
\definecolor{bargreen}{RGB}{0,128,0}
\definecolor{barred}{RGB}{255,110,100} %FF6961


\author{Breno Soares Alves}

\title{Controle PID aplicado a um sistema de navegação de aeronave de asa fixa}
      {PID control applied to fixed-wing aircraft navigation system}

\tipotrabalho{\tci}

\curso{\cc}

\orientador{Luciano Lores Caimi}

\palavraschave{aeromodelo, navegação, pid, controle, sistema}
              {aircraft, navigation, pid, control, system}

%----------------------------------------------------------------
% A capa é inserida automaticamente. Por isso não é necessário
% chamar \maketitle
%----------------------------------------------------------------
\begin{document}

%----------------------------------------------------------------
% Ficha catalográfica (OBRIGATÓRIO para TCC II)
%  Você pode gerar a ficha neste link também: 
%   https://ficha.uffs.edu.br/
%----------------------------------------------------------------
\fichacatalografica

%----------------------------------------------------------------
% Folha de aprovação (OBRIGATÓRIO para TCC II)
%   Primeiro parâmetro é a data da defesa
%   Demais parâmetros são os membros da banca (pelo menos 2 nomes)
%   Não precisa passar o nome do Orientador
%----------------------------------------------------------------
%\folhadeaprovacao{23/07/2023}{Fulano de Tal - UFFS}{Beltrano de Outro - UFFS}{}


%----------------------------------------------------------------
% Depois da folha de aprovação vem a dedicatória e a epígrafe.
%----------------------------------------------------------------
%\dedicatoria{Dedico este trabalho a meus pais.}

%\epigrafe{The art of simplicity is a puzzle of complexity.}
%         {Douglas Horton}

%----------------------------------------------------------------
% Também dá para fazer as duas na mesma página:
%----------------------------------------------------------------
%\dedigrafe{Dedico este trabalho a meus pais.}
%          {The art of simplicity is a puzzle of complexity.}
%          {Douglas Horton}

%----------------------------------------------------------------
% A seguir, a página de agradecimentos (OPCIONAL):
%----------------------------------------------------------------
%\begin{agradecimentos}
%À lorem ipsum, dolor sit amet consetetur sadipscing elitr sed diam
%nonumy eirmod tempor. invidunt ut labore et dolore magna aliquyam

%À erad sed, diam voluptua at vero, eos et accusam et justo duo
%dolores et ea rebum stet clita.

%À kasd gubergren, no sea. takimata sanctus est lorem ipsum dolor sit
%amet lorem ipsum dolor sit amet. consetetur sadipscing elitr sed

%À diam nonumy, eirmod tempor, invidunt ut labore et dolore magna
%aliquyam erat sed diam voluptua at.
%\end{agradecimentos}

%----------------------------------------------------------------
% Resumo (OBRIGATÓRIO)
%  As palavras chave são informadas no preambulo lá emcima (\palavraschave)
%----------------------------------------------------------------
\begin{resumo}
Seu resumo em português aqui. lorem ipsum dolor sit amet
consetetur sadipscing elitr sed diam nonumy eirmod tempor invidunt
ut labore et dolore magna aliquyam erat sed diam voluptua at vero
eos et accusam et justo duo dolores et ea rebum stet clita.  kasd
gubergren no sea takimata sanctus est lorem ipsum dolor sit amet
lorem ipsum dolor sit amet consetetur sadipscing elitr sed diam
nonumy eirmod tempor invidunt ut labore et dolore magna aliquyam
erat sed diam voluptua at.
\end{resumo}

%----------------------------------------------------------------
% Abstract (OBRIGATÓRIO)
% %  As keywords são informadas no preambulo lá emcima (\palavraschave)
%----------------------------------------------------------------
\begin{abstract}
Your abstract in English here. lorem ipsum dolor sit amet
consetetur sadipscing elitr sed diam nonumy eirmod tempor invidunt
ut labore et dolore magna aliquyam erat sed diam voluptua at vero
eos et accusam et justo duo dolores et ea rebum stet clita kasd
gubergren no sea takimata sanctus est lorem ipsum dolor sit amet
lorem ipsum dolor sit amet consetetur sadipscing elitr sed diam
nonumy eirmod tempor invidunt ut labore et dolore magna aliquyam
erat sed diam voluptua at
\end{abstract}

%----------------------------------------------------------------
% Listas e sumário, nessa ordem. Somente o sumário é obrigatório,
% portanto, comente as outras listas, caso sejam desnecessárias.
%----------------------------------------------------------------
%\listoffigures       % Lista de figuras      (OPCIONAL)
%\listoftables        % Lista de tabelas      (OPCIONAL)
%\listofalgorithms    % Lista de algoritmos   (OPCIONAL)
%\listofacronyms      % Lista de siglas       (OPCIONAL)
%\listofsymbols       % Lista de símbolos     (OPCIONAL)
\tableofcontents     % Sumário               (OBRIGATÓRIO)

%----------------------------------------------------------------
% Aqui começa o desenvolvimento do trabalho. Para uma melhor
% organização do documento, separe-o em arquivos,
% um para cada capítulo. Para isso, utilize o comando \include,
% como mostrado abaixo.
%----------------------------------------------------------------
\chapter{Introduction} % Contextualização e problema
% UAvs and their applications
% Control algorithms and PID (context, history and limitations)
% Problem (How to achieve great satisfaction level of reliability in PID model applied to fixed-wing UAVs)

\section{Motivation}
\label{sec:motivation}

Nowadays, the presence of automatic and autonomous systems in our daily lives are increasing exponentially. The presence of self-adapt models in many missions with high environmental changing rate is crucial for the task success helping humans to make decisions or increase situation awareness. These solutions are recognizable in modern aviation primarily in military branch. As integration between humans and machines becomes vital in many areas, the fragility of human life is highlighted as a limitation to mission achievement.

Let us get the actual reality of fighter jets, these machines are designed to perform high G's maneuvers and defying gravity being capable of reach an average maximum of 12 Gs in material stress without permanent damage (here i need to put some reference about), but the flight systems are limited to a maximum of 9 Gs because that value is the upper-bound for a well trained pilot sustain for a few seconds. \cite{WHINNERY2013}

The fragility essence of life does not just limit the machines potential in combat but also in transportation like the additional weight because of the life support systems necessary to maintain humans for long periods, like oxygen generation, seat belts, instrumentation, etc.

Considering these restrictions, the deployment of drones and Unmanned Aerial Vehicles (UAVs) became crucial, initially in military and over the recent decades they are being used in precision agriculture, surveillance and search and rescue without risking human lives and great cost-benefit in all areas. Drones and UAVs are capable of do the dull, dirty, dangerous and covert jobs equally or greater than humans with smaller size and without psychological effects depending only on technology improvement. \cite{AUSTINUAS2010}
 
 The deployment of automatic aerial vehicles in tasks of any nature faces the initial challenge of how to make a system capable of achieving straight leveled flight even with environmental degradation caused by crosswinds, turbulence, rain or thunderstorms. With the advance of mechanical automation, control models and miniaturization of electronic components, the assembly of systems with a high level of reliability in their behavior are possible and capable of transporting humans.

 Today, those systems have become more complex and difficult for a single person to conceptualize from scratch. Algorithms that control an Airbus A320 or Boeing 737 are gigantic and meticulous with high complex mathematics, physics and control models with many layers of redundancy to ensure stable, secure and pleasant flight for passengers and flight crew. Control systems that use algorithms like Linear Quadratic Regulator, adaptive PID and others require refined calculations and high precision models to ensure their reliability which means a lot of effort studying and tuning the system, but this is impractical for low critical missions or leisure purposes.

 The predecessor of those complex and modern control methods is the Proportional Integral Derivative model with your beginning around 1922 with Nicholas Minorsky for ship steering initially, but in the 1960s became the standard in analog autopilots for flying vehicles. The main concept of PID model concerns around three core variables, proportional, integral and derivative ones, where each of these is a constant. The PID was not overtaken by more recent control algorithms like the LQR, simply because of its simplicity and easier implementation over the modern ones.

This research addresses the challenge of creating a reliable autonomous navigation system for a fixed-wing UAV, specifically focusing on waypoint-based missions. Such missions, for example, transporting a medical payload from point A to be delivered via parachute at point B, or conducting systematic aerial photography for cartographic or surveillance purposes—demand a high degree of navigational precision and flight stability. While commercial off-the-shelf (COTS) flight controllers that perform these functions are readily available, they often function as "black boxes," offering limited customizability and little insight into their internal control logic.  
The core problem this work seeks to solve is: How can a PID-based control system be implemented from first principles on a custom-designed flight control unit to achieve a high level of reliability for waypoint-based navigation using the Global Navigation Satellite System (GNSS)?
The justification for this project lies in its educational and practical value. By designing the hardware and developing the software from the ground up, a deep and transparent understanding of the entire autonomous flight system is achieved. This approach provides the flexibility to tailor the system to unique airframes or mission requirements, a feat often difficult or impossible with proprietary COTS systems.

\section{Objectives}
\subsection{General Objectives}

The main goal of this work is to develop a complete navigation system for a self-made aircraft, featuring a custom PID flight controller, telemetry, and data logging — in order to evaluate whether it can achieve a high level of reliability when following predefined GNSS-based flight plans.

\subsection{Specific Objectives}

\begin{itemize}
    \item \textbf{Design and fabricate the hardware and airframe:} Develop the electronic architecture and printed circuit boards using CAD tools, while simultaneously designing and manufacturing the custom airframe through 3D printing. The resulting prototype will serve as the physical platform for testing the navigation and control system.
    
    \item \textbf{Implement manual radio control:} Perform initial flights using a traditional radio control interface to understand the aerodynamic behavior, stability, and response of the aircraft. This stage provides essential reference data for controller tuning.
    
    \item \textbf{Develop and tune a PID-based attitude control model:} Implement and adjust a custom PID controller to maintain stable pitch, roll, and yaw dynamics. During this stage, the control algorithm will operate under pilot supervision to ensure smooth transition toward autonomous flight.
    
    \item \textbf{Integrate GNSS-based waypoint navigation:} Extend the system capabilities by enabling autonomous navigation through predefined flight plans. The aircraft must be able to read, store, and execute waypoint sequences with controlled accuracy and position tolerance.
    
    \item \textbf{Implement telemetry and ground control interface:} Develop a ground station capable of bidirectional communication for sending commands, updating flight plans, monitoring telemetry, and triggering safety procedures such as “return to base.” 
    
    \item \textbf{Conduct autonomous flight testing and evaluation:} Execute a full mission flight from point A to point B under real environmental conditions. The objective is to assess system autonomy, control stability, and reliability by completing a surveillance task such as capturing images from a designated target location.
\end{itemize}

%\chapter{Motivation}    % Justificativa

\chapter{Theoretical Framework}

\section{Fundaments and core concepts}

\subsection{Fixed-Wing Aircraft Control}

Fixed-wing aircraft are aerial vehicles whose flight depends on rigid aerodynamic surfaces (airfoils), unlike rotary-wing aircraft that rely on rotating blades to produce thrust and lift.

The fundamental principle of flight is governed by Newton's Second Law, which relates the net force acting on a body to its acceleration:

\begin{equation}
    \vec{F} = m\vec{a}
\end{equation}

where $\vec{F}$ is the resultant force vector, \textit{m} is the aircraft's mass, and $\vec{a}$ is the linear acceleration.

During flight, four main forces act on the aircraft: lift, weight, thrust, and drag. Lift is generated perpendicular to the airflow over the wings and results from a pressure difference between the upper and lower surface of the airfoil, explained by Bernoulli's principle and Newton's Third Law. According to Bernoulli's equation, the sum of static, dynamic pressure, and gravitational potential energy along a streamline is constant:

\begin{equation}
    p + \frac{1}{2} \rho v^2 + \rho g h = constant
\end{equation}

In many aerodynamic cases — like airflow over a wing — the altitude \textit{h}doesn’t change much, so the gravitational term can be neglected, giving the simplified form:

\begin{equation}
    p + \frac{1}{2} \rho v^2 = constant
\end{equation}

However the Bernoulli's principle are impractical for predicting lift in real-world scenarios due to factors like viscosity, turbulence, and three-dimensional flow effects. Therefore, an empirical approach using dimensionless coefficients is often employed. The lift force \textit{L} can be expressed as:

\begin{equation}
    L = C_L \frac{1}{2} \rho v^2 S
\end{equation}

where $C_L$ is the lift coefficient, \textit{$\rho$} is the air density, \textit{v} is the velocity of the aircraft relative to the air, and \textit{S} is the wing area. The lift coefficient $C_L$ is determined experimentally and depends on factors such as the angle of attack, airfoil shape, and Reynolds number.

Drag is the aerodynamic force that opposes the aircraft's motion through the air caused by friction and pressure differences. It can be expressed similarly to lift:

\begin{equation}
    D = C_D \frac{1}{2} \rho v^2 S
\end{equation}

where $C_D$ is the drag coefficient, which also depends on factors like shape, surface roughness, and flow conditions.

Thrust is generated by the aircraft's engines to overcome drag and propel the aircraft forward. The weight ($W = mg$) is the force due to gravity acting downward on the aircraft's mass.

The equilibrium of these forces determines the aircraft's flight conditions, such as steady level flight, climbing, or descending. Control surfaces like ailerons, elevators, and rudders manipulate these forces to change the aircraft's attitude and trajectory.

\subsection{Aircraft Axes and Surfaces}

\section{Related Papers}
\chapter{Metodology}

The project development was divided into three unique areas:

\section{Engineering}

This section refers to all steps related to assembling, crafting and constructing the test platform for electronics and controlling areas. The final result is the single-engine, high-wing, fixed-gear, electric fixed-wing aircraft.

\begin{enumerate}
    \item \textbf{Structural Design:} Development of the airframe geometry, wings, and fuselage dimensions using CAD tools. This step ensures aerodynamic efficiency, mechanical integrity, and sufficient space for electronics integration. \\
    \textit{Expected result:} A digital 3D model validated through basic aerodynamic simulations.
    
    \item \textbf{3D Printing and Material Processing:} Fabrication of the aircraft’s structural components using 3D printing technology with suitable polymers such as PLA or PETG. This method allows for rapid prototyping, precise geometry replication, and easy design iteration while maintaining low production costs. \\
    \textit{Expected result:} High-accuracy printed airframe components ready for assembly and mechanical fitting.

    \item \textbf{Assembly and Integration:} Installation of propulsion, battery, and structural components, integrating the electronics housing and control surfaces. This step verifies fit, alignment, and accessibility for maintenance. \\
    \textit{Expected result:} A fully assembled aircraft prepared for system testing and calibration.

    \item \textbf{Ground Testing:} Execution of static and taxi tests to evaluate motor performance, mechanical alignment, and balance. This reduces risks before flight operations. \\
    \textit{Expected result:} Verified mechanical functionality and readiness for electronic integration and flight testing.
\end{enumerate}

\section{Electronics}

The platform for applying control algorithms to airplane flight surfaces is the self-made printed circuit board capable of high-speed processing and real-time scheduling with embedded sensors and modular capabilities.

\begin{enumerate}
    \item \textbf{PCB Design and Fabrication:} Design of a custom printed circuit board integrating two microcontrollers — one dedicated to flight control and sensor data processing, and another handling communication, telemetry, and auxiliary functions. This distributed architecture enhances system reliability, parallel task execution, and modular software development. \\
    \textit{Expected result:} A functional dual-microcontroller PCB providing robust computation and communication capabilities for flight control.

    \item \textbf{Sensor Integration:} Incorporation of IMUs, GPS, airspeed, and barometric sensors to provide precise data for state estimation. Proper calibration and filtering are critical for reliable control performance. \\
    \textit{Expected result:} Synchronized and calibrated sensor data streams available for real-time processing.

    \item \textbf{Communication System Setup:} Implementation of telemetry and radio control links for remote data acquisition and flight management. This ensures real-time monitoring and operator control. \\
    \textit{Expected result:} Reliable bidirectional communication between the aircraft and ground station.

    \item \textbf{Power and Safety Validation:} Verification of power distribution, current limits, and protection mechanisms (fuses, voltage regulators). This step ensures safe operation and prevents component damage. \\
    \textit{Expected result:} A stable power system supporting all onboard electronics under flight conditions.
\end{enumerate}

\section{Controlling}

The airplane navigation logic like sensors readings, filtering, data-flow and control model is available in the FCU firmware to manage state estimation and surfaces output to maintain stable flight.

\begin{enumerate}
    \item \textbf{Firmware Development:} Implementation of embedded software for sensor acquisition, control loop execution, and actuator commands. This is the core of the autonomous flight control system. \\
    \textit{Expected result:} Operational firmware running on the PCB, capable of stabilizing the aircraft in flight.

    \item \textbf{State Estimation and Filtering:} Application of sensor fusion algorithms (e.g., complementary or Kalman filters) to obtain accurate attitude and position estimates. Reliable state estimation is essential for robust control. \\
    \textit{Expected result:} Accurate and stable estimation of the aircraft’s orientation and velocity in real time.

    \item \textbf{Control Algorithm Implementation:} Development and tuning of PID or model-based control laws for pitch, roll, yaw, and altitude control. This ensures smooth and responsive flight performance. \\
    \textit{Expected result:} Stable flight dynamics with minimal oscillations and precise trajectory following.

    \item \textbf{Flight Testing and Validation:} Execution of controlled flight tests to validate firmware performance and tuning. Data from onboard logs are analyzed to refine models and control gains. \\
    \textit{Expected result:} Verified flight stability and reliable autonomous or assisted control performance.
\end{enumerate}

\section{Chronogram}

\begin{table}[ht]
    \begin{center}
    
    \begin{ganttchart}[
    x unit = 0.44cm,
    y unit title=0.5cm,
    y unit chart=0.5cm,
    vgrid,hgrid, 
    title label anchor/.style={below=-1.6ex},
    title left shift=.05,
    title right shift=-.05,
    title height=1,
    progress label text={},
    bar height=0.7,
    group right shift=0,
    group top shift=.6,
    group height=.3,bar/.append style={fill=bargreen},
    bar incomplete/.append style={fill=barred},
    group incomplete/.append style={draw=black,fill=none}]{1}{32}
    %labels
    \gantttitle{2024}{8} \gantttitle{2025}{24}\\
    \gantttitle{Nov}{4} 
    \gantttitle{Dez}{4} 
    \gantttitle{Jan}{4} 
    \gantttitle{Fev}{4} 
    \gantttitle{Mar}{4} 
    \gantttitle{Abr}{4}
    \gantttitle{Mai}{4}
    \gantttitle{Jun}{4} \\
    %tasks
    \ganttbar[progress=62]{Etapa 1}{1}{8} \\
    \ganttbar[progress=0]{Etapa 2}{7}{10} \\ %// mais pra lá a partir daqui 
    \ganttbar[progress=0]{Etapa 3}{10}{15} \\
    \ganttbar[progress=0]{Etapa 4}{13}{19} \\
    \ganttbar[progress=0]{Etapa 5}{18}{25} \\ %//maior
    \ganttbar[progress=0]{Etapa 6}{23}{30} \\ %//maior
    \ganttbar[progress=0]{Etapa 7}{26}{32} \\ 
    \ganttbar[progress=0]{Etapa 8}{23}{32} 
    %relations  
    %\ganttlink{elem0}{elem1} 
    %\ganttlink{elem0}{elem3} 
    %\ganttlink{elem1}{elem2} 
    %\ganttlink{elem3}{elem4} 
    %\ganttlink{elem1}{elem5} 
    %\ganttlink{elem3}{elem5} 
    %\ganttlink{elem2}{elem6} 
    %\ganttlink{elem3}{elem6} 
    \end{ganttchart}
    \end{center}
    
    \caption{Cronograma}
\label{tab:cronograma2}
\end{table}

%----------------------------------------------------------------
% Aqui vai a bibliografia. 
% Existem 2 estilos de citação definidos ao instanciar o 
% package abnt2tex: use
% 'alf' para citações do tipo [Abc+] ou [XYZ] (em ordem
% alfabética na bibliografia);
% 'num' para citações
% numéricas do tipo [1], [20], etc., em ordem de referência.
%----------------------------------------------------------------
\bibliography{bibliography}


%----------------------------------------------------------------
% Após \appendix, se iniciam os capítulos de Apêndice, com
% numeração alfabética.
%----------------------------------------------------------------
\appendix
\chapter{Meu primeiro apêndice}
\chapter{My second appendix}

%----------------------------------------------------------------
% Aqui vão os "capítulos" de anexos. Cada anexo deve
% ser considerado um capítulo.
%----------------------------------------------------------------
\anexos
\chapter{Meu primeiro anexo}
\chapter{My second attachment}

% E aqui (para a felicidade de todos) termina o documento.
\end{document}
